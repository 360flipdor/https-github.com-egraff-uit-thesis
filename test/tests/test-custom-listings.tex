\documentclass{uit-thesis-test}

\makeglossaries

\usepackage{filecontents}

% Note: this is only used for test-purposes (create temporary file samplecode.c
% with the given contents, so we can test \lstinput<...> macros)
\begin{filecontents*}{samplecode.c}
#include <stdio.h>

int main(void)
{
  printf("Hello world!\n");
  return 0;
}
\end{filecontents*}

% Global options (apply to all listings)
\lstset{%
  captionpos=b,
  numbers=left,
  backgroundcolor=\color[rgb]{0.9,0.9,1},
  basicstyle={\footnotesize}
}

% Create custom "codesnippet" environment + \lstinputcodesnippet macro
% with separate counting and styling
\newcustomlstenvironment{codesnippet}{locodesnippet}{Code Snippet}[1][]
  {%
    \lstset{%
      language=C,
      frame=tb,
      captionpos=t,
      backgroundcolor=\color[rgb]{1,1,0.9}
    }%
    \lstset{#1}% Accept arguments from environment / lstinput macro
    \csname lst@SetFirstNumber\endcsname%
  }
  {%
    \csname lst@SaveFirstNumber\endcsname%
  }

% Create \listofcodesnippet macro
\newlistof{codesnippet}{locodesnippet}{List of Code Snippets}

% Create custom "algorithm" environment + \lstinputalgorithm macro
% with separate counting and styling
\newcustomlstenvironment{algorithm}{loalgorithm}{Algorithm}[1][]
  {%
    \lstset{%
      language=C,
      frame=tblr,
      captionpos=t,
      backgroundcolor=\color[rgb]{0.9,1,0.9}
    }%
    \lstset{#1}% Accept arguments from environment / lstinput macro
    \csname lst@SetFirstNumber\endcsname%
  }
  {%
    \csname lst@SaveFirstNumber\endcsname%
  }

% Demonstrate separate caption styling
\captionsetup[codesnippet]{box=colorbox,boxcolor={yellow}}
\captionsetup[lstlisting]{box=colorbox,boxcolor={cyan},skip=-1pt}

\begin{document}

\frontmatter
\tableofcontents
\listof{lstlisting}{List of Listings}
\listofcodesnippet
\listof{algorithm}{List of Algorithms}

\mainmatter

\chapter{Chapter}

\begin{lstlisting}[caption={Test 1},label={listing:test1},name={test}]
Line 1
Line 2
\end{lstlisting}

\begin{lstlisting}[caption={Test 2},label={listing:test2},name={test}]
Line 3
Line 4
\end{lstlisting}

Referencing normal listings: \autoref{listing:test1}, \autoref{listing:test2}.

\begin{codesnippet}[caption={First snippet},label={snippet:first},name={hello}]
#include <stdio.h>
\end{codesnippet}

\begin{algorithm}[caption={First algorithm},label={alg:first},name={myalg}]
A
B
\end{algorithm}

\begin{codesnippet}[caption={Second snippet},label={snippet:second},name={hello}]
int main(void)
{
  printf("Hello world!\n");
  return 0;
}
\end{codesnippet}

Referencing snippets: \autoref{snippet:first}, \autoref{snippet:second}.

\begin{algorithm}[caption={Second algorithm},label={alg:second},name={myalg}]
C
\end{algorithm}

Referencing algorithms: \autoref{alg:first}, \autoref{alg:second}.

\chapter{Input}

\lstinputcodesnippet[caption={this is my caption},label={snippet:input}]{samplecode.c}

Referencing snippet: \autoref{snippet:input}.

\end{document}
